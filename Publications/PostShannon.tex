%%%% IEEE DEFAULTS

\documentclass[conference]{IEEEtran}
\IEEEoverridecommandlockouts
% The preceding line is only needed to identify funding in the first footnote. If that is unneeded, please comment it out.
%\usepackage{cite} % should not be used with natbib
%\usepackage{amsmath,amssymb,amsfonts}
%\usepackage{algorithmic}
\usepackage{graphicx}
\usepackage{enumitem}
\usepackage{subcaption}
\usepackage{adjustbox}
\usepackage{relsize}
\usepackage[font=small]{caption}
%\usepackage{textcomp}
%foot note
\usepackage[bottom]{footmisc}
\usepackage{xcolor}
%\def\BibTeX{{\rm B\kern-.05em{\sc i\kern-.025em b}\kern-.08em
%    T\kern-.1667em\lower.7ex\hbox{E}\kern-.125emX}}

%%%% USER SETTINGS

\PassOptionsToPackage{hyphens}{url}
\usepackage[hidelinks]{hyperref}
\hypersetup{hidelinks}% doppelt hält besser
%\usepackage[acronym]{glossaries} % https://www.overleaf.com/learn/latex/Glossaries#Acronyms
\usepackage[acronym]{glossaries}
\makeglossaries
\input{acronym}

\usepackage{hyperref}

\usepackage{csquotes}
\usepackage[numbers]{natbib}
\bibliographystyle{IEEEtranN}
\renewcommand{\bibfont}{\footnotesize} % for IEEE bibfont size
\usepackage[per-mode=symbol-or-fraction]{siunitx}
%\usepackage{flushend}
%\bstctlcite{bibliography:BSTcontrol}
\usepackage{array}
\renewcommand\arraystretch{1.25}
\usepackage{booktabs}

\renewcommand{\figurename}{Figure}
\newcommand{\cgn}[1]{\textcolor{blue}{(GN): #1}}

% todo notes:
\setlength {\marginparwidth}{2cm}
\usepackage{todonotes}
\presetkeys{todonotes}{inline}{}

\usepackage{pgf}
\usepackage{tikz}
\usetikzlibrary{shapes.geometric, arrows}
\tikzstyle{startstop} = [rectangle, rounded corners, minimum width=3cm, minimum height=1cm,text centered, draw=black, fill=red!30]
\tikzstyle{io} = [trapezium, trapezium left angle=70, trapezium right angle=110, minimum width=3cm, minimum height=1cm, text centered, draw=black, fill=blue!30]
\tikzstyle{process} = [rectangle, minimum width=3cm, minimum height=1cm, text centered, draw=black, fill=orange!30]
\tikzstyle{decision} = [diamond, minimum width=3cm, minimum height=1cm, text centered, draw=black, fill=green!30]
\usetikzlibrary{backgrounds}
\usetikzlibrary{arrows}
\usetikzlibrary{shapes,shapes.geometric,shapes.misc}
\usetikzlibrary{decorations.pathreplacing,calc,shadows.blur,shapes}
\usetikzlibrary{positioning}
\usetikzlibrary{shapes.geometric}
\tikzstyle{tikzfig}=[baseline=-0.25em,scale=0.5]
\usepackage{color}
\usetikzlibrary{positioning}

\begin{document}

\title{Need for Speed: Hardware-based Radios for virtualized Radio Access Networks}
\author{
	\IEEEauthorblockN{
		Javier Acevedo\IEEEauthorrefmark{1},
		Marian Ulbricht\IEEEauthorrefmark{1},
		Florian Grabs\IEEEauthorrefmark{1},
		Andreas Igno Grohmann\IEEEauthorrefmark{1},
		Giang T. Nguyen\IEEEauthorrefmark{2}\IEEEauthorrefmark{4},\\ % needed to line break because of too many authors :)
		Patrick Seeling\IEEEauthorrefmark{3},
		and
		Frank H. P. Fitzek\IEEEauthorrefmark{1}\IEEEauthorrefmark{4}
	}
	\IEEEauthorblockA{
		\IEEEauthorrefmark{1} Deutsche Telekom Chair of Communication Networks, TU Dresden
		}
	\IEEEauthorblockA{
		\IEEEauthorrefmark{2} Chair of Haptic Communication Systems, TU Dresden
		}
	\IEEEauthorblockA{
		\IEEEauthorrefmark{3} Department of Computer Science, Central Michigan University
		}
	\IEEEauthorblockA{
		\IEEEauthorrefmark{4} Centre for Tactile Internet with Human-in-the-Loop (CeTI)
		}
	E-mails: \{firstname.lastname\}@tu-dresden.de\IEEEauthorrefmark{1}\IEEEauthorrefmark{2} ;	\{firstname.lastname\}@cmich.edu\IEEEauthorrefmark{3}
}

\maketitle

% enable page numbers, remove for final version
\thispagestyle{plain}
\pagestyle{plain}

\begingroup\renewcommand\thefootnote{\textsection}

\endgroup

\input{sections/00_abstract}

\begin{IEEEkeywords}
Hardware Accelerator, Haptic Communication, Tactile Internet, Cryptographic Algorithms, Human-machine Interaction, Ultra-low Latency, High-performance Computing, Network Security Functions. 
\end{IEEEkeywords}


\input{sections/10_introduction}
\input{sections/20_technical_background}
\input{sections/30_related_work}
\input{sections/40_methodology}
\input{sections/50_evaluation}
\input{sections/60_outlook}
\input{sections/70_conclusions}

%\input{70_conclusions}

\section*{Acknowledgement}
This work was funded by the German Research Foundation (DFG, Deutsche Forschungsgemeinschaft) as part of Germany’s Excellence Strategy – EXC 2050/1 – Project ID 390696704 – Cluster of Excellence “Centre for Tactile Internet with Human-in-the-Loop” (CeTI) of Technische Universität Dresden.

We also would like to thank Juan Cabrera and Riccardo Bassoli for their valuable insights during the implementation of the results presented in this work.

%Print bibliography with IEEE Tran style
\bibliography{bibliography}

\end{document}
